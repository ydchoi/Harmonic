\documentclass{article} % For LaTeX2e
\usepackage{nips14submit_e,times}
\usepackage{amsmath}
\usepackage{amsthm}
\usepackage{amssymb}
\usepackage{mathtools}
\usepackage{hyperref}
\usepackage{url}
\usepackage{algorithm}
\usepackage[noend]{algpseudocode}
%\documentstyle[nips14submit_09,times,art10]{article} % For LaTeX 2.09

\usepackage{mathrsfs}
\usepackage{graphicx}
\usepackage{caption}
\usepackage{subcaption}

\def\eQb#1\eQe{\begin{eqnarray*}#1\end{eqnarray*}}
\def\aB#1\aE{\begin{align*}#1\end{align*}}
\def\eQnb#1\eQne{\begin{align}#1\end{align}}
\providecommand{\e}[1]{\ensuremath{\times 10^{#1}}}
\providecommand{\pb}[0]{\pagebreak}

\newcommand{\E}{\mathrm{E}}
\newcommand{\Var}{\mathrm{Var}}
\newcommand{\Cov}{\mathrm{Cov}}

\def\Qb#1\Qe{\begin{question}#1\end{question}}
\def\Sb#1\Se{\begin{solution}#1\end{solution}}

\usepackage{scalerel,stackengine}
\stackMath
\newcommand\reallywidehat[1]{%
\savestack{\tmpbox}{\stretchto{%
  \scaleto{%
    \scalerel*[\widthof{\ensuremath{#1}}]{\kern-.6pt\bigwedge\kern-.6pt}%
    {\rule[-\textheight/2]{1ex}{\textheight}}%WIDTH-LIMITED BIG WEDGE
  }{\textheight}% 
}{0.5ex}}%
\stackon[1pt]{#1}{\tmpbox}%
}

\newenvironment{claim}[1]{\par\noindent\underline{Claim:}\space#1}{}
\newtheoremstyle{quest}{\topsep}{\topsep}{}{}{\bfseries}{}{ }{\thmname{#1}\thmnote{ #3}.}
\theoremstyle{quest}
\newtheorem*{definition}{Definition}
\newtheorem*{theorem}{Theorem}
\newtheorem*{lemma}{Lemma}
\newtheorem*{question}{Question}
\newtheorem*{preposition}{Preposition}
\newtheorem*{exercise}{Exercise}
\newtheorem*{challengeproblem}{Challenge Problem}
\newtheorem*{solution}{Solution}
\newtheorem*{remark}{Remark}
\usepackage{verbatimbox}
\usepackage{listings}
\title{Harmonic Analysis:  \\
Problem Set V}


\author{
Youngduck Choi \\
CIMS \\
New York University\\
\texttt{yc1104@nyu.edu} \\
}


% The \author macro works with any number of authors. There are two commands
% used to separate the names and addresses of multiple authors: \And and \AND.
%
% Using \And between authors leaves it to \LaTeX{} to determine where to break
% the lines. Using \AND forces a linebreak at that point. So, if \LaTeX{}
% puts 3 of 4 authors names on the first line, and the last on the second
% line, try using \AND instead of \And before the third author name.

\newcommand{\fix}{\marginpar{FIX}}
\newcommand{\new}{\marginpar{NEW}}

\nipsfinalcopy % Uncomment for camera-ready version

\begin{document}


\maketitle

\begin{abstract}
This work contains solutions to the problem set IV
of Harmonic Analysis 2016 at Courant Institute of Mathematical Sciences.
\end{abstract}

\bigskip

\begin{question}[1]
\hfill
\begin{figure}[h!]
  \centering
    \includegraphics[width=0.8\textwidth]{HA-5-1.png}
\end{figure}
\end{question}
\begin{solution} \hfill \\
\textbf{(a)}
Let $p < 2$.
Assume that there is no uniform constant $c_p > 0$ such that $||\hat{f}||_{p'} \geq 
c_p ||f||_{p}$ for all $f \in S$. Since Fourier transform is a bijective mapping
from $S$ to $S$, and, for $f \in S$, $||\hat{\hat{f}}||_{p'} = ||f||_{p'}$ (Fourier transform
composed twice on the Schwarz space is a parity operator, which does not change
any $L^p$ norm for $f \in S$), substituting $\hat{f}$
in the position of $f$ implies that there is no uniform constant $c_p > 0$ such that 
\eQb
||f||_{p'} &\geq& c_p ||\hat{f}||_{p}, 
\eQe
for any $f \in S$. Since $S \subset L^p$, we conclude that the given statement is sufficient.

\bigskip

\textbf{(b)} With $u = \sqrt{i}x$, it follows that
\eQb
\int e^{ix^2} = \dfrac{1}{\sqrt{i}} \int e^{u^2} du = (\dfrac{1}{\sqrt{2}}
- i \dfrac{1}{\sqrt{2}} )\text{erf} (u) + C, 
\eQe
where the last equality holds with the well-known error function. Since $|\text{erf}(z)| \leq 1$ 
for any $z \in \mathbb{C}$, it follows that
\eQb
|\int_{a}^{b} e^{ix^2} dx| &=& 
|(\dfrac{1}{\sqrt{2}} - i\dfrac{1}{\sqrt{2}})|
|\text{erf}(b) - \text{erf}(a)| \\ 
&\leq&  2|(\dfrac{1}{\sqrt{2}} - i\dfrac{1}{\sqrt{2}})| \leq C,
\eQe
for some absolute constant $C$.

\bigskip

\textbf{(c)} By definition of Fourier transform, we have
\eQb
\hat{f_{\lambda}}(\xi) &=& \int_{\mathbb{R}} f_{\lambda}(x) e^{-2\pi i x \xi} dx 
=\int_{\mathbb{R}} e^{-\pi x^2}e^{-\pi i \lambda x^2} e^{-2\pi i x \xi} dx, 
\eQe
By integration by parts, it follows that
\eQb
\int_{-N}^{N} e^{-\pi x^2}e^{-\pi i \lambda x^2} e^{-2\pi i x \xi} dx &=&  
\eQe
It follows that, for some absolute constant $C$, with any $\xi \in \mathbb{R}$,
\eQb
|\hat{f_{\lambda}}(\xi)| \leq C\lambda^{-\frac{1}{2}},
\eQe
so
\eQb
||\hat{f}_{\lambda}||_{\infty} \leq C\lambda^{-\frac{1}{2}}, 
\eQe
as required.

\bigskip

\textbf{(d)} By generalized Holder's inequality with $n = 2$ and $\theta = \frac{2}{p'}$, we obtain
\eQb
||\hat{f}_{\lambda}||_{p'} &\leq& ||\hat{f}_{\lambda}||_{2}^{\frac{2}{p'}} 
||\hat{f}_{\lambda}||_{\infty}^{1-\frac{2}{p'}}.
\eQe

As $|e^{-\pi i \lambda x^2}| = 1$ for any $x \in \mathbb{R}$, and $||f||_{2} = ||\hat{f}||_{2}$ for $f 
\in L^2$,  
we see that
\eQb
||\hat{f}_{\lambda}||_{2} &=& ||f_{\lambda}||_{2} = 
(\int_{\mathbb{R}} |e^{-\pi x^2}e^{-\pi i \lambda x^2}|^p)^{\frac{1}{p}} 
= ||e^{-\pi x^2}||_{p} = C, 
\eQe
for some absolute constant $C$. Therefore, by $(c)$ and the above equality, it follows that
\eQb
||\hat{f}_{\lambda}||_{p'} &\leq& ||\hat{f}_{\lambda}||_{2}^{\frac{2}{p'}} 
||\hat{f}_{\lambda}||_{\infty}^{1-\frac{2}{p'}} \leq C\lambda^{\frac{1}{p'}-\frac{1}{2}},
\eQe
for some absolute constant $C$. Now, we deduce $(a)$. Similarly, we have
\eQb
||f_{\lambda}||_{p} &=& (\int_{\mathbb{R}} |e^{-\pi x^2}e^{-\pi i \lambda x^2}|^p)^{\frac{1}{p}} 
= ||e^{-\pi x^2}||_{p} = \delta, 
\eQe
for some constant $\delta > 0$, independent of $\lambda$.
 Let $c_p > 0$ be given. Then, for $\lambda > 0$ sufficiently small, it follows that
\eQb
||\hat{f}_{\lambda}||_{p'} \leq C\lambda^{\frac{1}{p'}-\frac{1}{2}} \leq c_p ||f_{\lambda}||_{p}, 
\eQe
where $C$ is some absolute constant. Therefore, there does not exist a uniform constant $c_p > 0$
such that $||\hat{f}||_{p'} \geq c_p ||f||_{p}$ for all $f \in S$.

\bigskip


\textbf{(e)} Let $1 \leq p < 2$.
Suppose for sake of contradiction that Fourier transform is a surjective 
map from $L^p$ to $L^{p'}$. As the Fourier 
transform is unique on $L^p$ for $1 \leq p < 2$, this implies that the Fourier transform is a 
bijective map from $L^p$ to  $L^{p'}$. Now, by the Open Mapping theorem, we have that
the Fourier transform is a bijective, open map from $L^p$ to $L^{p'}$. Then,
the inverse map of the Fourier transform is well-defined and continuous from $L^{p'}$
to $L^p$. This is a contradiction to $(a)$.

\end{solution}

\newpage

\begin{question}[2]
\hfill
\begin{figure}[h!]
  \centering
    \includegraphics[width=0.8\textwidth]{HA-5-2.png}
\end{figure}
\end{question}
\begin{solution} \hfill \\
\textbf{(a)}
Let $f \in B^{1}_{\Lambda}$, and
$X$ be the characteristic function of $[-\Lambda, \Lambda]$. As $f \in L^{1}$, applying 
the inversion to $\hat{f} = \hat{f}X$ gives
\eQb
f &=& f * \check{X}. 
\eQe
Arising from the fact that the Fourier transform is a bijective mapping on $S$ to $S$,
$S$ is in $L^p$ for any $1 \leq p \leq \infty$, and $X \in S$, 
it follows that $\check{X} \in S$, so $\check{X} \in L^p$ for any $p \geq 1$.
Therefore for any $1 \leq p \leq \infty$, by Young's inequality for convolution (to be exact, it
is the form recorded in Schlag pg.74) , we obtain
\eQb
||f||_{p} = ||f * \check{X}||_{p} \leq ||f||_{1} ||\check{X}||_{p} < \infty.
\eQe
Hence $f \in L^p$ for any $1 \leq p \leq \infty$, so we have shown that $B^1_{\Lambda} \subset 
L^p$ for $1 \leq p \leq \infty$.

\bigskip

\textbf{(b)} 
As $f \in L^1$, by the inversion formula, we have, for any $x \in \mathbb{R}$(Lebesgue identifiable point),
\eQb
f(x) = \int_{-\Lambda}^{\Lambda} \hat{f}(\xi)e^{2\pi i x \xi } dx.
\eQe


\bigskip

\textbf{(c)} Set $\varphi_{\Lambda}(\xi) = \Lambda \varphi(\Lambda x)$. Then, by a change of variable
$y = \Lambda x$, we obtain
\eQb
\hat{\varphi_{\Lambda}}(\xi) &=& \int \varphi(\Lambda x) e^{i\frac{\xi}{\Lambda} \Lambda x} d\Lambda x 
= \int \varphi(y) e^{i\frac{\xi}{\Lambda y}} dy = \hat{\varphi}(\frac{\xi}{\Lambda}), 
\eQe
so 
\eQb
\hat{\varphi_{\Lambda}}(\xi) = 1 \> \> \text{ for } |\xi| \leq \Lambda.
\eQe
Therefore, by the noted observation and the inversion, it follows that
\eQb
\widehat{f * \varphi_{\Lambda}} = \hat{f} \hat{\varphi_{\Lambda}} = \hat{f} \> \text{ and } \>
f * \varphi_{\Lambda} = f.
\eQe
Differentiating the last identity $m$ times, which is justified by $(b)$, and choosing to
differentiate $\varphi_{\Lambda}$ term in the convolution yields
\eQb
f^{(m)} = f * \varphi_{\Lambda}^{(m)}.
\eQe
By the inequality employed in part $(a)$, it follows that
\eQb
||f^{(m)}||_{p} &=& ||f * \varphi_{\Lambda}^{(m)}||_{p} \leq ||\varphi_{\Lambda}^{(m)}||_{1} ||f||_{p}
\leq (c\Lambda)^m||f||_{p}, \\
\eQe
for some absolute constant $c$, as the Schwartz class is closed under differentiation and the last
inequality follows from the definition of Schwartz class. 

\bigskip

\textbf{(d)} From classical complex analysis, we have the following theorem.

\begin{theorem}
Let $F(z,\xi)$ be defined for $(z,\xi) \in \Omega \times [-M,M]$ where $\Omega$ is an
open set in $\mathbb{C}$. Suppose $F$ satisfies the following properties: (i) 
$F(z,\xi)$ is holomorphic in $z$ for each $\xi$. (ii) $F$ is continuous on $\Omega \times [-M,M]$. 
Then, $f(z) = \int_{[-M,M]} \hat{f}(\xi) e^{2\pi i \xi z} d\xi$ is holomorphic.  
\end{theorem} 

The proof can be found in pg. 56 of Stein's Complex Analysis. To invoke the theorem, 
we take $\Omega = \mathbb{C}$,
$M = \Lambda$, and $F(z,\xi) = \hat{f}(\xi)e^{2\pi i \xi z}$. For any $\xi \in [-\Lambda, \Lambda]$,
$\hat{f}(\xi)e^{2\pi i \xi z}$ is continuous, so $(i)$ is satisfied. As $\hat{f}$ is uniformly continuous,
$(ii)$ is also satisfied, so we conclude that $f$ is entire. 
\hfill $\qed$

\end{solution}

\newpage

\begin{question}[3]
\hfill
\begin{figure}[h!]
  \centering
    \includegraphics[width=0.8\textwidth]{HA-5-3.png}
\end{figure}
\end{question}
\begin{solution} \hfill \\

\textbf{(a)} By definition of ONS, we have
\eQb
\{ e_n \varphi(x)\}_{m \in \mathbb{Z}} \> \text{ is an ONS} \iff 
\int e_n \varphi \overline{e_m \varphi}(x) dx = 0 
\eQe

\textbf{(b)}
As $\varphi \in L^2$, we have $\hat{\varphi} \in L^2$. From $(a)$, it follows that
\eQb
\sum_{m \in \mathbb{Z}} |\hat{\varphi}(\xi-m)|^2 = 1 \> \text{ a.e. } \xi
\iff 
\{ e^{2\pi i m \xi} \hat{\varphi}(\xi) \}_{m \in \mathbb{Z}} \> \text{ is an ONS}.
\eQe
As the Fourier transform is an isometry on $L^2$, we have
\eQb
\{ \varphi(\xi -m)\}_{m \in \mathbb{Z}} \> \text{ is an ONS} 
\iff
\{ \widehat{\varphi(\xi-m)}\}_{m \in \mathbb{Z}} \> \text{ is an ONS},
\eQe 
which through the identity $\widehat{\varphi(\xi-m)} = \hat{\varphi}(\xi)e^{-2\pi i m\xi}$ 
and the first equivalence implies
\eQb
\sum_{m \in \mathbb{Z}} |\hat{\varphi}(\xi - m)|^2 = 1 \> \text{ a.e. } \xi 
\iff
\{\varphi(\xi -m )\}_{m \in \mathbb{Z}} \> \text{ is an ONS} 
\eQe
as required. \hfill $\qed$
\end{solution}

\newpage 

\begin{question}[4]
\hfill
\begin{figure}[h!]
  \centering
    \includegraphics[width=0.8\textwidth]{HA-5-4.png}
\end{figure}
\end{question}
\begin{solution} \hfill \\
\textbf{(a)} We restate the results for $B^2_{\Lambda}$ instead of $B^1_{\Lambda}$,
but without changing the lower bound on $p$ from $1$ to $2$. 

\bigskip

\textbf{(b)} Let $\{f_n\}$ be a sequence in $B_{\Lambda}^2$ such that it converges to
some $f \in L^2$. As $||\hat{g}||_2 = ||g||_2$ for any $g \in L^2$, it follows that
$\{\hat{f_n}\}$ converges to $\hat{f}$ in $L^2$. Since the property of support being contained
in compact set
persists through $L^2$ limit (this trivially can be shown using a proof by contradiction), 
it follows that $\text{supp}\hat{f} \subset [-\Lambda,\Lambda]$,
so $f \in B_{\Lambda}^2$. Hence, $B_{\Lambda}^2$ is closed. 

\smallskip

We now show that $B_{\Lambda}^2$ is invariant under translations. Fix $h \in \mathbb{R}$, and
let $\eta_{h}$ be defined by $(\eta_{h}f)(x) = f(x-h)$ on $L^2$. Now, consider the modulation
operator $m_{h}$ to be defined by $(m_{h}f)(x) = e^{2\pi ih x}f(x)$. Then, it follows 
that, for $f \in L^2$, and $\xi \in \mathbb{R}$, 
\eQb
\widehat{\eta_{h}f}(\xi) &=& \int f(x-h)e^{-2\pi i \xi x} dx = (m_{-h}\hat{f})(\xi),
\eQe
so 
\eQb
\text{supp}\> \> \widehat{\eta_{h} f} = \text{supp}\> \> {m_{-h}}\hat{f} = \text{supp}
\>\> \hat{f}, 
\eQe
as one can trivially see that a support of a function is an invariant property under 
the modulation operator.
Therefore, we have shown that the translation operator is invariant on $B_{\Lambda}^2$.


\bigskip 

\textbf{(c)} 
Now, $f \in B_{\Lambda}^{1}$ can be expressed as
\eQb
f &=& \sum_{n \in \mathbb{Z}} <f,\text{sinc}(\cdot - n)>\text{sinc}(\cdot -n)
\eQe    

\bigskip

\textbf{(d)} By Cauchy-Scwhwarz, it follows that
\eQb
\sum_{n \in \mathbb{Z}}|<f,\text{sinc}(\cdot-n)>||\text{sinc}(t-n)| &\leq& (\sum_{n \in \mathbb{Z}}
|<f,\text{sinc}(\cdot -n)>|^2)^{\frac{1}{2}} (\sum_{n \in \mathbb{Z}} \text{sinc}^2(t-n))^{\frac{1}{2}}. 
\eQe
In view of Parseval's theorem, we see that the first sum on the RHS converges, so
it suffices to show that the $\text{sinc}$ sum is uniformly bounded, but 
this can be shown with the comparison test with $\sum_{n \in \mathbb{Z}} \dfrac{1}{n^2}$. 
Therefore, the sum is uniformly bounded for all $t \in \mathbb{R}$, so the convergence is uniform.

\end{solution}

\end{document}
