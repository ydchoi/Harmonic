\documentclass{article} % For LaTeX2e
\usepackage{nips14submit_e,times}
\usepackage{amsmath}
\usepackage{amsthm}
\usepackage{amssymb}
\usepackage{mathtools}
\usepackage{hyperref}
\usepackage{url}
\usepackage{algorithm}
\usepackage[noend]{algpseudocode}
%\documentstyle[nips14submit_09,times,art10]{article} % For LaTeX 2.09

\usepackage{mathrsfs}
\usepackage{graphicx}
\usepackage{caption}
\usepackage{subcaption}

\def\eQb#1\eQe{\begin{eqnarray*}#1\end{eqnarray*}}
\def\aB#1\aE{\begin{align*}#1\end{align*}}
\def\eQnb#1\eQne{\begin{align}#1\end{align}}
\providecommand{\e}[1]{\ensuremath{\times 10^{#1}}}
\providecommand{\pb}[0]{\pagebreak}

\newcommand{\E}{\mathrm{E}}
\newcommand{\Var}{\mathrm{Var}}
\newcommand{\Cov}{\mathrm{Cov}}

\def\Qb#1\Qe{\begin{question}#1\end{question}}
\def\Sb#1\Se{\begin{solution}#1\end{solution}}

\usepackage{scalerel,stackengine}
\stackMath
\newcommand\reallywidehat[1]{%
\savestack{\tmpbox}{\stretchto{%
  \scaleto{%
    \scalerel*[\widthof{\ensuremath{#1}}]{\kern-.6pt\bigwedge\kern-.6pt}%
    {\rule[-\textheight/2]{1ex}{\textheight}}%WIDTH-LIMITED BIG WEDGE
  }{\textheight}% 
}{0.5ex}}%
\stackon[1pt]{#1}{\tmpbox}%
}

\newenvironment{claim}[1]{\par\noindent\underline{Claim:}\space#1}{}
\newtheoremstyle{quest}{\topsep}{\topsep}{}{}{\bfseries}{}{ }{\thmname{#1}\thmnote{ #3}.}
\theoremstyle{quest}
\newtheorem*{definition}{Definition}
\newtheorem*{theorem}{Theorem}
\newtheorem*{lemma}{Lemma}
\newtheorem*{question}{Question}
\newtheorem*{preposition}{Preposition}
\newtheorem*{exercise}{Exercise}
\newtheorem*{challengeproblem}{Challenge Problem}
\newtheorem*{solution}{Solution}
\newtheorem*{remark}{Remark}
\usepackage{verbatimbox}
\usepackage{listings}
\title{Harmonic Analysis:  \\
Final Exam}


\author{
Youngduck Choi \\
CIMS \\
New York University\\
\texttt{yc1104@nyu.edu} \\
}


% The \author macro works with any number of authors. There are two commands
% used to separate the names and addresses of multiple authors: \And and \AND.
%
% Using \And between authors leaves it to \LaTeX{} to determine where to break
% the lines. Using \AND forces a linebreak at that point. So, if \LaTeX{}
% puts 3 of 4 authors names on the first line, and the last on the second
% line, try using \AND instead of \And before the third author name.

\newcommand{\fix}{\marginpar{FIX}}
\newcommand{\new}{\marginpar{NEW}}

\nipsfinalcopy % Uncomment for camera-ready version

\begin{document}


\maketitle

\begin{abstract}
This work contains a solution to the Final Exam
of Harmonic Analysis 2016 at Courant Institute of Mathematical Sciences.
\end{abstract}

\bigskip

\begin{question}[1]
\hfill
\begin{figure}[h!]
  \centering
    \includegraphics[width=0.8\textwidth]{HA-f-1.png}
\end{figure}
\end{question}

\newpage 

\begin{solution} \hfill \\
\textbf{(a)} Let $P$ be a trig polynomial defined on $\mathbb{T}$ of degree $N$, i.e. 
\eQb
P &=& \sum_{|k| \leq N} a_k e_k,
\eQe 
where $a_k$s are the complex coefficients. 
Suppose $|k| \leq N$. We trivially know that $\hat{P}(k) = a_k$. We compute
\eQb
\widehat{p_k} &=& \sum_{n \in \mathbb{Z}_{2N+1}} p_n e^{\frac{-2\pi i n k}{2N+1}} 
= \sum_{n \in \mathbb{Z}_{2N+1}} \dfrac{1}{2N+1} P(\dfrac{n}{2N + 1}) e^{-\frac{2\pi i n k}{2N+1}} \\
&=& \dfrac{1}{2N+1}\sum_{n \in \mathbb{Z}_{2N+1}} \left( 
\sum_{|l| \leq N} a_l e_{l}(\dfrac{n}{2N+1}) \right) 
e^{-\frac{2\pi i n k}{2N+1}}  
= \dfrac{1}{2N+1}\sum_{n \in \mathbb{Z}_{2N+1}} \left( 
\sum_{|l| \leq N} a_l e^{\frac{2\pi i l n}{2N+1}} \right) 
e^{-\frac{2\pi i n k}{2N+1}} \\
&=& \dfrac{1}{2N+1}\sum_{n \in \mathbb{Z}_{2N+1}} \left( 
\sum_{|l| \leq N} a_l e^{\frac{2\pi i (l-k) n}{2N+1}} \right) 
= \dfrac{1}{2N+1} \sum_{|l| \leq N} \left( a_l \sum_{n \in \mathbb{Z}_{2N+1}} e^{\frac{2\pi i (l-k)n}{2N+1}}
\right) \\
&=& \dfrac{1}{2N+1} \sum_{|l| \leq N; l \neq k} \left( 
a_l \sum_{n \in \mathbb{Z}_{2N+1}} e^{\frac{2\pi i (l-k)n}{2N+1}} 
\right)
+ \dfrac{1}{2N+1} \sum_{|l| \leq N; l = k} \left( 
a_l \sum_{n \in \mathbb{Z}_{2N+1}} e^{\frac{2\pi i (l-k)n}{2N+1}} \right) \\
&=& 0 + \dfrac{1}{2N+1}(2N+1) a_k = a_k,
\eQe
as the sum of any $N$-th root of unity is zero, thereby forcing the first term of the second last equation 
to be $0$. 
For $|k| > N$, we trivially see that $\widehat{p}_k = 0$, and $\hat{P}(k) = 0$ as well.
\hfill $\qed$
\end{solution}

\bigskip

\begin{question}[1-2]
\hfill
\begin{figure}[h!]
  \centering
    \includegraphics[width=0.8\textwidth]{HA-f-1-2.png}
\end{figure}
\end{question}
\begin{solution} \hfill \\
\textbf{(b)} Let $ f \in C(\mathbb{T})$, and $P$ be a trig polynomial of degree $N$. 
Define as before $f \triangleq (f_n)_{n \in \mathbb{Z}}$ to be the $(2N+1)$-periodic sequence given by 
\eQb
f_n \triangleq \dfrac{1}{2N+1}F(\dfrac{n}{2N+1}).
\eQe
With $F = (F- P) + P$ and $f = (f-p) + p$, by the result of $(a)$, for any $|k| \leq N$, we see that
\eQb \label{eq1}
|\widehat{f_k} - \hat{F}_k| &=& |\reallywidehat{\{(f-p) + p\}}_k - 
\reallywidehat{(F-P)+P}(k)| 
= |\widehat{(f-p)}_k - \reallywidehat{(F-P)}(k) + \hat{p}_k - \hat{P}(k)| \nonumber \\
&\leq&  
|\widehat{(f-p)}_k| + |\reallywidehat{(F-P)}(k)| + | \hat{p}_k - \hat{P}(k)| 
=
|\widehat{(f-p)}_k| + |\reallywidehat{(F-P)}(k)|. \>\> (1)
\eQe
Now, in view of \eqref{eq1}, and the fact that $N$ is finite, it suffices to show that for any $|k| \leq N$,
\eQb
|\widehat{(f-p)}_k| \leq ||F-P||_{\infty} \> \text{and} \> |\reallywidehat{(F-P)}(k)| \leq ||F-P||_{\infty}.
\eQe
Now, for $|k| \leq N$, the first inequality follows, as
\eQb
|\widehat{(f-p)}_k| &=& |\sum_{n \in \mathbb{Z}_{2N+1}} (f-p)_{n} e^{\frac{-2\pi i n k}{2N+1}}| \\
&=& |\dfrac{1}{2N+1} \sum_{n \in \mathbb{Z}_{2N+1}} (F(\dfrac{n}{2N+1}) - P(\dfrac{n}{2N+1}) )
e^{\frac{-2\pi i n k}{2N+1}}| \\
&\leq& \dfrac{1}{2N+1} \sum_{n \in \mathbb{Z}_{2N+1}} |F(\dfrac{n}{2N+1}) - P(\dfrac{n}{2N+1})| = ||F-P||_{\infty}.  
\eQe

Likewise, for $|k| \leq N$, the second inequality, as $F-P \in C(\mathbb{T})$, and 
\eQb
|\reallywidehat{F-P}(k)| &=& \left| \int_{\mathbb{T}} (F-P)(t)e^{-ikt} dt \right|  \leq 
\int_{\mathbb{T}} |(F-P)(t)| dt \\
&=& ||F-P||_{1} \leq ||F-P||_{\infty}.
\eQe
Therefore, we have that for any trig polynomial of degree $N$, $P$, 
\eQb
\max_{|k| \leq N} | \hat{f}_{k} - \hat{F}(k)| &\leq& 2||F- P||_{\infty},
\eQe
as required. 

\bigskip

\textbf{(c)}
To begin with, in the video, we can see a particular physical interpretation of the mathematical
setup we have. $F$ is the square wave and the $f$ are the dots drawn on the square wave, which
corresponds to the regular samples of $F$. Now, setting the rocker arms, which controls
the coefficients of the cosine waves, to the values of $f_n$, we get the DFT by reading the
output wave's discrete positions. 
The important distinction here is that,
with the modeling assumption that the given data is an even function, Michelson managed to 
perform both the synthesis and analysis with the "syntehsis" (adding the cosine waves,
multiplied by the appropriate coefficients) as the DFT and inversion of DFT are the same
operations when restricted to cosines. This could be regarded further as a truncated DTFT as mentioned.
 
\bigskip

\textbf{(d)} Here, we want to see how close the discrete Fourier coefficients gets to the continuous 
Fourier coefficient as a function of $N$, which mathematically corresponds to the cardinality of the
finite Abelian group under consideration. This is precisely the question of convergence of Riemann sum
to an integral. Here, we are given that the function is piecewise continuous and linear, so by a well-known
result which says the rate of convergence Riemann sum to Riemann integral of $x^d$ is $\Theta(n^{-(1+d)})$,
we see that with our assumption the rate of convergence is $\Theta(\dfrac{1}{n^2})$. So, for $1000$,
we approximately have that the difference is bounded above by $\dfrac{1}{1000^2}$.  


\end{solution}

\newpage

\begin{question}[2]
\hfill
\begin{figure}[h!]
  \centering
    \includegraphics[width=0.8\textwidth]{HA-f-2.png}
\end{figure}
\end{question}
\begin{solution} \hfill \\
\textbf{(a)}
We first prove the equivalence for $\{e_n\}$. 
Fix $ h \in \mathbb{T}$.
Suppose that, for some $\xi \in l^{\infty}(\mathbb{Z})$, we have
\eQb
Te_n &=& \sum_{k \in \mathbb{Z}} \xi_{k} \hat{e_n}(k) e_k, 
\eQe
for all $n$. As $\hat{e_n}(k) = 1$, if $k = n$ and $0$ otherwise, it follows that, for all $n$,
\eQb
Te_n &=& \xi_{n} e_n,
\eQe
and, by linearity of $T$,
\eQb
T(\tau_{h}(e_n)) &=& T(e^{2\pi i n(\cdot-h)}) = e^{2\pi i n h} T(e_n) 
= e^{-2\pi i n h} \xi_n e_n = \tau_h(T(e_n)).
\eQe
Conversely, assume that $T(\tau_h e_n) = \tau_hT(e_n)$, for all $n$. 
Set
\eQb
\{\xi_n \} = \{\widehat{Te_n}(n)\}.
\eQe
We claim that the above construction gives the desired property. Firstly, $\{\xi_n  \}$
is bounded, as $Te_n \in L^2 \subset L^1$ (we are on $\mathbb{T}$), so by Riemann-Lebesgue lemma,
\eQb
\xi_n = \widehat{Te_n}(n) \to 0,
\eQe 
as $n \to \infty$, and $\{ \xi_n \} \in l^{\infty}(\mathbb{Z})$. Now, it remains to be shown that,
for all $n$,
\eQb
Te_n = \sum_{k \in \mathbb{Z}} \widehat{Te_k}(k) \widehat{e_n}(k)e_k. 
\eQe
As $T$ is bounded on $L^2$, it follows that
\eQb
Te_n = \sum_{k \in \mathbb{Z}} \widehat{Te_n}(k) e_k,
\eQe
and by the fact that $T$ commutes with the translation we get
\eQb
\sum_{k \in \mathbb{Z}} \widehat{Te_n}(k) e^{2\pi i (k-n)h} e^{2\pi i k(\cdot -h)} = \sum_{k \in \mathbb{Z}}
\widehat{Te_n}(k) e^{2\pi i k(\cdot-h)},
\eQe
so
\eQb
\widehat{Te_n}(k) e^{2\pi i(k-n)h} = \widehat{Te_n}(k). 
\eQe
Now, the equality holds for all $h \in \mathbb{T}$, so we have that $\widehat{Te_n}(k) = 0$ for any $k \neq n$,
and hence,
\eQb
Te_n = \widehat{Te_n}(n) e_n,
\eQe
as required.

\smallskip

Now, we argue that the equivalence can be extended to any $f \in L^2$. Suppose that the operator
commutes with the translation operator for any $e_n$. Then, it follows that, for any $f \in L^2$,
\eQb
T(\tau_h f) &=& T(\tau_h \sum_{n \in \mathbb{Z}} \hat{f}(n)e_n) = \sum_{n \in \mathbb{Z}} \hat{f}(n)
T(\tau_h e_n) \\
&=& \sum_{n \in \mathbb{Z}} \hat{f}(n) \tau_h(Te_n) = \tau_h(T f). 
\eQe
The converse holds trivially. Now, suppose that the operator has the given description with respect to
$\{e_n\}$. It follows that, for any $f \in L^2$,
\eQb
Tf &=& T(\sum_{n \in \mathbb{Z}} \hat{f}(n) e_n ) = \sum_{n \in \mathbb{Z}} \hat{f}(n) T(e_n) = 
\sum_{n \in \mathbb{Z}} \xi_n \hat{f}(n)  e_n. 
\eQe
The converse in this case also holds trivially. Thus, we have shown the desired equivalence. 

\smallskip

We now argue that $||T||_{2 \to 2} = ||\xi||_{\infty}$. By Parseval, we obtain
\eQb
||T||_{2 \to 2} &=& \sup_{||f||_{2} = 1} ||Tf||_{2} = \sup_{||f||_{2} =1} 
(\sum_{n \in \mathbb{Z}} |\xi_n \hat{f}(n)|^2)^{\frac{1}{2}}  \\ &\leq& \sup_{||f||_{2} = 1} 
||\xi||_{\infty} (\sum_{n \in \mathbb{Z}}|\hat{f}(n)|^2 )^{\frac{1}{2}} = ||\xi||_{\infty}. 
\eQe 
Now, we construct $f \in L^2$, such that $||Tf|| = ||\xi||_{\infty}$. If $||\xi||_{\infty} = 0$,
then, the equality immediately follows. If $||\xi||_{\infty} > 0$, as $\{\xi\}_{\infty}$ was 
shown to converge to $0$, via Riemann-Lebesgue lemma, there exists a finite index $N$ such that
$|\xi_N| = ||\xi||_{\infty}$. Then, we have that
\eQb
||T e_N|| = |\xi_N|,  
\eQe
so the equality holds.
\bigskip

\textbf{(b)}
First, suppose that $T f = f * \mu $ for some $\mu \in M(\mathbb{T})$. It follows that
\eQb
T(\tau_h f) &=& (\tau_h f) * \mu =  \tau_h (f * \mu) = \tau_h(T f).
\eQe
Now, conversely, suppose that $T$ commutes with translations.  
Consider the Fejer kernel, denoted by $\{ K_n \}$. 
Firstly, observe that as the Fejer kernel is an approximate identity, it follows that
$\{T K_n\}$ are bounded on $M(\mathbb{T})$. By the weak-* compactness of $M(\mathbb{T})$,
we can extract a subsequence $\{ K_{n_l} \}$ 
such that $K_{n_l} \to \mu$ weakly, for some $\mu \in M(\mathbb{T})$.
 For notational convenience, we
relabel the subsequence as $\{ K_n \}$. Now, assume that $T$ commutes 
with translations. It follows that, for any $f \in C(\mathbb{T})$, and $x \in \mathbb{T}$, 
\eQb
f * \mu (x) &=& \lim_{n \to \infty} \int_{\mathbb{T}} f(t)T(K_n) (x - t) dt 
= \lim_{n \to \infty} \int_{\mathbb{T}} f(t)T(K_n(x - t)) dt, \\
&=& \lim_{n \to \infty} \int_{\mathbb{T}} T(f(t)(K_n) (x - t)) dt, 
= \lim_{n \to \infty} T\left(\int_{\mathbb{T}} f(t) K_n(x-t) dt \right),
\eQe
which via boundedness of $T$ and the fact that for any $f \in C(\mathbb{T})$, 
\eQb
\lim_{n \to \infty} (f*K_n) (x) = f(x),
\eQe
implies that
\eQb
f* \mu(x)&=& T \left( \lim_{n \to \infty} \int_{\mathbb{T}} f(t) K_n(x-t) dt \right) = Tf(x),
\eQe  
as required. Now, as $C(\mathbb{T})$ is dense in $L^1(\mathbb{T})$, via standard density argument,
it follows that, for any $f \in L^1(\mathbb{T})$,
\eQb
f* \mu &=& Tf.
\eQe 

\smallskip

By Young's inequality for convolution, it follows that
\eQb
||T||_{1\to 1} = \sup_{||f|| = 1} ||f * \mu||_{1} \leq \sup_{||f|| = 1 } ||f||_{1} ||\mu||_{M(\mathbb{T})}
= ||\mu||_{M(\mathbb{T})}.
\eQe
We now construct $f \in L^1$ such that $||Tf||_{1} = ||\mu||_{M(\mathbb{T})}$. As $||u|| < \infty$,
we can choose, a sequence of partition, whose sum of absolute value of the measures of each part, converges
to the total variation norm. Now, consider the sequence of functions defined by by the sum of characteristic
functions of the corresponding $n$th partition, denoted by $f_n$. By construction, it follows that
\eQb
||T f_n||_1 = ||f_n * \mu|| \to ||\mu||,
\eQe
as $n \to \infty$, and the equality holds. \hfill $\qed$

\end{solution}

\bigskip

\begin{question}[3]
\hfill
\begin{figure}[h!]
  \centering
    \includegraphics[width=0.8\textwidth]{HA-f-3.png}
\end{figure}
\end{question}
\begin{solution} \hfill \\
\textbf{(a)} We fix the index in the definition of $m$ to start at $0$ and end at $n-1$.
 Suppose $m$ is a monotonic step function given by 
\eQb
m &=& \sum_{i=0}^{n-1} c_i X_{[\alpha_{i}, \alpha_{i+1})}.
\eQe
As $T_{[0,\infty)} = (iH + \dfrac{1}{2}I)$ (this fact was discussed in class), we have that
\eQb
T_{[a,\infty)}f &=& e^{2\pi i a x} iHf(x) + \dfrac{1}{2} f(x),
\eQe
so
\eQb
T_{m}f &=& T_{\sum_{i=0}^{n-1} c_i X_{[\alpha_i, \alpha_{i+1}]}} 
= c_0 I + \sum_{i=0}^{n-2} (c_{i+1} - c_i) T_{[\alpha_{i+1},\infty)} \\
&=& c_0 I + 
 \sum_{i=0}^{n-2} (c_{i+1}-c_i)
(e^{2\pi i \alpha_{i+1}} iHf + \dfrac{1}{2}f) .
\eQe
Now, by the triangle inequality, and the fact that the Hilbert Transform is bounded on $L^p$ 
(recall that the bound constant is dependent on $p$) for 
$1 < p < \infty$, we obtain
\eQb
||T_m f||_{p} &\leq& |c_0|||f||_{p} + \sum_{i=0}^{n-2} |c_{i+1}
- c_i|(||Hf||_{p} + \dfrac{1}{2}||f||_{p}) \\
&\leq& C_p (|c_0| + \sum_{i=0}^{n-2} |c_{i+1} - c_i|) ||f||_p
\leq C_p (||m||_{\infty} + |m|_{TV}) ||f||_{p}, \\
\eQe
for some constant $C_p$, depended on $p$, so
\eQb
||T_m||_{p\to p} \leqslant_{p} (||m||_{\infty} + |m|_{TV}),
\eQe
as required.

\bigskip

\textbf{(b)}
Let $m$ be a function of bounded variation. As having a bounded variation implies being bounded, it follows that
\eQb
||m||_{\infty} < \infty. 
\eQe
Now, as $m$ is measurable,  we can 
choose a sequence of monotonic step functions that converge pointwise to $m$ such that 
$|m_n| \leq |m|$ for all $n$. Therefore, by DCT, for any $f \in L^p$, with $1 < p < \infty$,
it follows that, for the pointwise limit,
\eQb
\lim_{n \to \infty} T_{m_n} f &=& \lim_{n \to \infty} \int_{\mathbb{R}} m_n(\xi) \hat{f}(\xi)  
e^{2\pi i \xi} d\xi 
= \int_{\mathbb{R}} m(\xi)\hat{f}(\xi) e^{2\pi i \xi} d\xi = T_m f. 
\eQe
Therefore, by Fatou 
\eQb
||T_m f||_{p} \leq \liminf_{n} ||T_{m_n} f||_{p},
\eQe
and by $(a)$, and the choice of $\{m_n\}$
\eQb
||T_m f||_{p} \leq \liminf_{n} ||T_{m_n} f||_{p} \leqslant_{p} (||m||_{\infty} + |m|_{TV}),
\eQe
as required. \hfill $\qed$
\end{solution}

\bigskip

\begin{question}[4]
\hfill
\begin{figure}[h!]
  \centering
    \includegraphics[width=0.8\textwidth]{HA-f-4.png}
\end{figure}
\end{question}
\begin{solution} \hfill \\

\textbf{(a)} Let $f \in L^1$. Taking the Fourier transform on both sides gives
\eQb
\hat{f} &=& \hat{f}\hat{f},
\eQe
which, with the continuity of $\hat{f}$, implies that
\eQb
\hat{f} = 0 \> \text{ a.e} \> &\text{ or }& \> \hat{f} = 1 \> \text{a.e}.
\eQe
As $\hat{f} = 1 \> \text{ a. e}$ contradicts the Riemann-Lebesgue lemma, it follows that
\eQb
\hat{f} &=& 0 \> \text{ a.e.},
\eQe
so by the inversion formula for $L^1$
\eQb
f &=& 0 \> \text{ a.e.},
\eQe 
as required.

\bigskip

\textbf{(b)} Let $f \in L^2$ such that $f = f * f$. By the same argument in the $L^1$ case, 
we have 
\eQb
\hat{f} = 0 \> \text{ a.e} \> &\text{ or }& \> \hat{f} = 1 \> \text{a.e}.
\eQe
Let $E_1 = \{ \hat{f} = 1\}$ and $E_0 = \{ \hat{f} = 0 \}$.
As $\hat{f} \in L^2$, it follows that 
\eQb
||\hat{f}||_{2} &=& (\int_{\mathbb{R}} |\hat{f}(\xi)|^2)^{\frac{1}{2}} = m(E_1)^{\frac{1}{2}} < \infty, 
\eQe
so
\eQb
m(E_1) < \infty,
\eQe
and, for any $p \geq 1$, 
\eQb
||\hat{f}||_{p} &=& (\int_{\mathbb{R}} |\hat{f}(\xi)|^p)^{\frac{1}{p}} = m(E_1)^{\frac{1}{p}} < \infty. 
\eQe
Therefore, we see that, for $1 \leq p \leq \infty$,
\eQb
\hat{f} \in L^p,
\eQe
as the infinity bound follows trivially.

\smallskip

Now, we first prove the uniform continuity of $f$. By the inversion formula for $L^2$,
we obtain
\eQb
f(x) &=& \int_{\mathbb{R}} \hat{f}(\xi)e^{2\pi i \xi x} d\xi = \int_{E_1} e^{2\pi i \xi x} d\xi. 
\eQe
Therefore, for any $\delta > 0$ and $x \in \mathbb{R}$, it follows that
\eQb
|f(x + \delta) - f(x)| &=& | \int_{E_1} e^{2\pi i \xi (x + \delta)}  - 
e^{2\pi i \xi x}| d\xi  
\leq  \int_{E_1} |e^{2\pi i \xi x }| |e^{2\pi i \xi \delta } - 1| d\xi \\
&\leq& 
\int_{E_1} |e^{2\pi i \xi \delta } - 1| d\xi. \\
\eQe
Observe that the last integral is independent of $x$, and the integrand tends to $0$,
as $\delta \to 0$. Therefore, we have shown that $f$ is uniformly continuous. 

\smallskip

We now argue that
$f \in L^p$ for $p \in [2,\infty]$. We employ Riesz-Thorin to the Fourier inversion operator. 
Since the Fourier inversion is bounded from $L^1$ to $L^{\infty}$ and from $L^2$ to $L^2$, by
Riesz-Thorin, we have that the inversion is bounded from $p$ to $q$ where $1 \leq p \leq 2$
and $q$ is the conjugate of $p$. In particular, for $2 \leq p \leq \infty$, we see that
\eQb
||f||_{p} &\leq& ||\hat{f}||_{q}, 
\eQe
where $q$ is again the conjugate of $p$.
As we have previously shown that $\hat{f} \in L^p$, for all $1 \leq p \leq \infty$, we are done.

\bigskip

\textbf{(c)} Let $f \in L^p$ such that $f = f*f$. As $p \in (1,2)$, it follows that $\hat{f} \in L^q$,
where $q$ is the conjugate of $p$.
As $\hat{f} \in L^q$, by the same argument from $(b)$, we have that $\hat{f} \in L^2$.
Therefore, we have shown that $\hat{f} \in L^2$, so $f \in L^2$, as the Fourier transform is an isometry on
$L^2$. \hfill $\qed$  

\end{solution}
\end{document}
