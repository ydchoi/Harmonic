\documentclass{article} % For LaTeX2e
\usepackage{nips14submit_e,times}
\usepackage{amsmath}
\usepackage{amsthm}
\usepackage{amssymb}
\usepackage{mathtools}
\usepackage{hyperref}
\usepackage{url}
\usepackage{algorithm}
\usepackage[noend]{algpseudocode}
%\documentstyle[nips14submit_09,times,art10]{article} % For LaTeX 2.09

\usepackage{mathrsfs}
\usepackage{graphicx}
\usepackage{caption}
\usepackage{subcaption}

\def\eQb#1\eQe{\begin{eqnarray*}#1\end{eqnarray*}}
\def\aB#1\aE{\begin{align*}#1\end{align*}}
\def\eQnb#1\eQne{\begin{align}#1\end{align}}
\providecommand{\e}[1]{\ensuremath{\times 10^{#1}}}
\providecommand{\pb}[0]{\pagebreak}

\newcommand{\E}{\mathrm{E}}
\newcommand{\Var}{\mathrm{Var}}
\newcommand{\Cov}{\mathrm{Cov}}

\def\Qb#1\Qe{\begin{question}#1\end{question}}
\def\Sb#1\Se{\begin{solution}#1\end{solution}}

\usepackage{scalerel,stackengine}
\stackMath
\newcommand\reallywidehat[1]{%
\savestack{\tmpbox}{\stretchto{%
  \scaleto{%
    \scalerel*[\widthof{\ensuremath{#1}}]{\kern-.6pt\bigwedge\kern-.6pt}%
    {\rule[-\textheight/2]{1ex}{\textheight}}%WIDTH-LIMITED BIG WEDGE
  }{\textheight}% 
}{0.5ex}}%
\stackon[1pt]{#1}{\tmpbox}%
}

\newenvironment{claim}[1]{\par\noindent\underline{Claim:}\space#1}{}
\newtheoremstyle{quest}{\topsep}{\topsep}{}{}{\bfseries}{}{ }{\thmname{#1}\thmnote{ #3}.}
\theoremstyle{quest}
\newtheorem*{definition}{Definition}
\newtheorem*{theorem}{Theorem}
\newtheorem*{lemma}{Lemma}
\newtheorem*{question}{Question}
\newtheorem*{preposition}{Preposition}
\newtheorem*{exercise}{Exercise}
\newtheorem*{challengeproblem}{Challenge Problem}
\newtheorem*{solution}{Solution}
\newtheorem*{remark}{Remark}
\usepackage{verbatimbox}
\usepackage{listings}
\title{Harmonic Analysis:  \\
Problem Set IV}


\author{
Youngduck Choi \\
CIMS \\
New York University\\
\texttt{yc1104@nyu.edu} \\
}


% The \author macro works with any number of authors. There are two commands
% used to separate the names and addresses of multiple authors: \And and \AND.
%
% Using \And between authors leaves it to \LaTeX{} to determine where to break
% the lines. Using \AND forces a linebreak at that point. So, if \LaTeX{}
% puts 3 of 4 authors names on the first line, and the last on the second
% line, try using \AND instead of \And before the third author name.

\newcommand{\fix}{\marginpar{FIX}}
\newcommand{\new}{\marginpar{NEW}}

\nipsfinalcopy % Uncomment for camera-ready version

\begin{document}


\maketitle

\begin{abstract}
This work contains solutions to the problem set IV
of Harmonic Analysis 2016 at Courant Institute of Mathematical Sciences.
\end{abstract}

\bigskip

\begin{question}[1]
\hfill
\begin{figure}[h!]
  \centering
    \includegraphics[width=0.8\textwidth]{HA-4-1.png}
\end{figure}
\end{question}
\begin{solution} \hfill \\
\textbf{(a)}
Firstly, note that $H$ is a bounded linear operator on the space of trig polynomials for any $p$
by assumption.
For any $f \in L^p$, by density of trig polynomials, there exists $\{f_n\}$ such that
$f_n \overset{L^p}{\to} f$. Observe that by linearity and boundedness of $H$ on the space
of trig polynomials, we have
\eQb
||Hf_n - Hf_m||_{p} &=& ||H(f_n - f_m)||_{p} \leq C_p||f_n - f_m||_{p},
\eQe  
where $C_p$ is the operator bound on the space of trig polynomials. By the above estimate, 
$\{H f_n\}$ is Cauchy in $L^p$, thus convergent by completeness of $L^p$. Therefore, we can
define $Hf$ as 
\eQb
Hf &=& \lim_{n \to \infty} H f_n.
\eQe
We first show that the extension is well-defined. Let $f \in L^{p}(\mathbb{T})$, and consider
$\{g_n\}$ and $\{ h_n\}$ trig polynomials such that they converge to $f$ in $L^p$. Analogous the
previous estimate, we have
\eQb
||Hg_n - Hh_n||_{p} &=& ||H(g_n - h_n)||_p \leq C_p||g_n - h_n|| \to 0,
\eQe 
as $n \to \infty$. Therefore, $\lim_{n\to \infty} Hg_n = \lim_{n \to \infty} H h_n$ and
the extension is well-defined. Similarly, it follows that the operator is linear and
bounded on the whole $L^p$. We now argue the uniqueness of the operator. 
Let $H_1, H_2$ be bounded linear operators on $L^{p}(\mathbb{T})$ that agree on the space
of trig polynomials. As the trig polynomials are dense in $L^p(\mathbb{T})$, we have that
$H_1 = H_2$ on the entire $L^{p}(\mathbb{T})$, by the continuity of $H_1$ and $H_2$.
Therefore, the defined operator is unique.  

\smallskip

Let $f \in L^{p}(\mathbb{T})$, and $\{p_n\}$ be trig polynomials such that $p_n \overset{L^p}{\to} f$,
as $n \to \infty$. Since
\eQb
\int_{\mathbb{T}} |f_n - f||e^{2\pi i n x}| dx \leq ||f_n - f||_{p},
\eQe
we have
\eQb
\hat{f_k}(n) \to \hat{f}(n) \text{ as} \> k \to \infty.
\eQe
It follows that
\eQb
\widehat{Hf}(n) &=& \int Hf e^{-2\pi n \theta} d\theta 
= \lim_{k \to \infty} \int Hf_k e^{-2\pi n \theta} d\theta \\
&=& \lim_{k \to \infty} \widehat{H f_k}(n) = -i\text{sgn}(n) \hat{f}(n).
\eQe

\bigskip

\textbf{(b)} Applying the above result twice to $H^2f$ yields
\eQb
\widehat{H^{2}f}(n) &=&  
\left\{
    \begin{array}{ll}
        -\hat{f}(n)  & \mbox{if } n \neq 0 \\
        0 & \mbox{if } n = 0, \\
    \end{array}
\right.
\eQe
Since,
\eQb
\widehat{-f+\hat{f}(0)}(n) &=&  
\left\{
    \begin{array}{ll}
        -\hat{f}(n)  & \mbox{if } n \neq 0 \\
        0 & \mbox{if } n = 0, \\
    \end{array}
\right.
\eQe
it follows that, for any $n \in \mathbb{Z}$, 
\eQb
\widehat{H^2 f}(n) = \widehat{-f + \hat{f}(0)}(n), 
\eQe
which by uniqueness of fourier coefficients implies
\eQb
H^2 f &=& -f + \hat{f}(0),
\eQe
as required.


 \hfill $\qed$
\end{solution}

\bigskip

\begin{question}[2]
\hfill
\begin{figure}[h!]
  \centering
    \includegraphics[width=0.8\textwidth]{HA-4-2.png}
\end{figure}
\end{question}
\begin{solution} \hfill \\
\textbf{(a)} By the given identity, we have
\eQb
(f+ iHf)^2 = f^2 - (Hf)^2 + i(2fHf),
\eQe
which by linearity of $H$ implies that 
\eQb
H(f^2 - (Hf)^2) &=& H((f + iHf)^2) - H(i2fHf). 
\eQe
Therefore, to prove $(4)$, it suffices to show that
\eQb
\reallywidehat{H((f+iHf)^2)}(n) &=& \reallywidehat{2(fHf + iH(fHf)}(n),
\eQe
for any $n \in \mathbb{Z}$. We compute
\eQb
f+iHf = \hat{f}(0) + 2\sum_{n > 0} \hat{f}(n)e^{2\pi i n \theta},
\eQe
and
\eQb
\widehat{f+iHf}(n) &=& 
\left\{
    \begin{array}{ll}
        2\hat{f}(n)  & \mbox{if } n > 0 \\
        \hat{f}(0) & \mbox{if } n = 0, \\
        0 & \mbox{if } n < 0
    \end{array}
\right.
\eQe
so
\eQb
\reallywidehat{(f+iHf)^2}(n) &=& \sum_{m \in \mathbb{Z}} 
\reallywidehat{f+iHf}(n-m)\reallywidehat{f+iHf}(m) \\
&=& 
\left\{
    \begin{array}{ll}
        \sum_{m=1}^{n}  4\hat{f}(n-m)\hat{f}(m) & \mbox{if } n > 0 \\
        \hat{f}(0)^2 & \mbox{if } n = 0, \\
        0 & \mbox{if } n < 0.
    \end{array}
\right.
\eQe
Thus
\eQb
\reallywidehat{H(f+iHf)}(n) &=& 
\left\{
    \begin{array}{ll}
         -4i\sum_{m=1}^{n} \hat{f}(n-m)\hat{f}(m) & \mbox{if } n > 0 \\
        0 & \mbox{if } n = 0, \\
        0 & \mbox{if } n < 0.
    \end{array}
\right.
\eQe
We again compute
\eQb
\widehat{fHf}(n) &=& \sum_{m \in \mathbb{Z}} \hat{f}(n-m)(-i)\text{sgn}(m)\hat{f}(m) \\
&=& -i\sum_{m \in \mathbb{Z}} \hat{f}(n-m)\hat{f}(m),
\eQe
so
\eQb
\reallywidehat{2(fHf+iH(fHf))}(n) &=& 
\left\{
    \begin{array}{ll}
         4\widehat{fHf}(n) = -4i\sum_{m=1}^{n}\hat{f}(n-m)\hat{f}(m)& \mbox{if } n > 0 \\
        0 & \mbox{if } n = 0, \\
        0 & \mbox{if } n < 0.
    \end{array}
\right.
\eQe
Therefore, we have 
\eQb
\reallywidehat{H(f+iHf)^2}(n) = 2\reallywidehat{fHf + iH(fHf)}(n),
\eQe
for any $n \in \mathbb{Z}$ as required. Now, from $(4)$ we deduce $(5)$. Applying $H$ to both sides
of $(4)$ yields
\eQb
2H(fHf) &=& H^2(f^2 - (Hf)^2) = -f^2 + (Hf)^2 + \reallywidehat{f^2 - (Hf)^2}(0).
\eQe 
Since 
\eQb
\reallywidehat{f^2 - (Hf)^2}(0) &=& \sum_{m \in \mathbb{Z}} \hat{f}(-m)\hat{f}(m) + 
\text{sgn}(-m)\text{sgn}(m)\hat{f}(-m)\hat{f}(m) \\
&=& \hat{f}(0)^2, 
\eQe
we obtain $(5)$ as desired. \hfill $\qed$

\bigskip

\textbf{(b)} Let $Z = f + ig$ be any complex-valued trig polynomial, where $f$ and $g$
are the real-valued trig polynomials. Observe that
\eQb
(HF)^2 = (Hf)^2 + 2iHfHg - (Hg)^2
\eQe
and 
\eQb
H(FHF) = H((f+ig)(Hf +iHg)) = H(fHf -gHg - i(fHg + gHf)).
\eQe
Therefore, it suffices to show that 
\eQb
HfHg = fg - \hat{f}(0)\hat{g}(0) + H(fHg + gHf).
\eQe
Expanding the RHS of the given identity gives
\eQb
4HfHg &=& (H(f+g)^2) - (H(f-g))^2 \\
&=& (f+g)^2 - (f-g)^2 + 2H((f-g)(Hf + Hg)) - 2H((f-g)(Hf-Hg)) \\ 
&-& (\hat{f}(0) + \hat{g}(0))^2 + (\hat{f}(0)^2 - \hat{g}(0))^2 \\
&=& 4fg - 4\hat{f}(0)\hat{g}(0) + 4H(fHg + gHf).
\eQe
Dividing the above identity by $4$ gives the desired identity as desired. \hfill $\qed$
\end{solution}

\bigskip

\begin{question}[3]
\hfill
\begin{figure}[h!]
  \centering
    \includegraphics[width=0.8\textwidth]{HA-4-3.png}
\end{figure}
\end{question}
\begin{solution} \hfill \\
Substituting the second identity from the problem 2 with, via Minkowski's inequality, we obtain
\eQb
||Hf||_{2p}^2 &=& ||Hf^2||_{p} \leq ||f^2||_{p}  + ||\hat{f}(0)^2||_{p} + ||2H(fHf)||_{p} \\
&\leq& ||f||_p^2 + |\hat{f}(0)|^2 + ||2H(fHf)||_{p},
\eQe
where we have simply used the fact that a constant function in $L^p(\mathbb{T})$ attains
the modulus of the value for the norm of any $p \in (1,\infty)$. 
Now, by monotonicity of $L^p(\mathbb{T})$ norms, it follows that
\eQb
|\hat{f}(0)|^2 &=& |\int_{\mathbb{T}} f(x) dx|^2 \leq (\int_{\mathbb{T}} |f(x)| dx)^2 
= ||f||_{1}^2 \leq ||f||_{2p}^2.
\eQe
Furthermore, by generalized Holder's inequality, we have
\eQb
||2H(fHf)||_{p} \leq 2C_p ||fHf||_{p} \leq 2C_p ||f||_{2p} ||Hf||_{2p}.
\eQe
Therefore, combining the two estimates gives
\eQb
||Hf||_{2p}^2 &\leq& 2||f||_{2p}^2 + 2C_p||f||_{2p} ||Hf||_{2p}, 
\eQe
as required. Since
\eQb
||Hf||_{2p}^2 - 2||f||_{2p}^2 - 2C_p||f||_{2p} ||Hf||_{2p} \leq 0, 
\eQe
utilizing the quadratic formula, we have
\eQb
||Hf||_{2p}^2 \leq \dfrac{2C_p||f||_{2p} + \sqrt{4C_p^2||f||_{2p}^2 + 2||Hf||_{2p}^2}}{2},
\eQe
which can be simplified to
\eQb
||Hf||_{2p} &\leq& C_p||f||_{2p} + \sqrt{C_p^2 + 2}||f||_{2p},
\eQe
which then implies 
\eQb
C_{2p} &\leq& C_p + \sqrt{C_p^2 + 2}, 
\eQe
as required.
As $C_2 = 1$, by the above inequality, we obtain that $C_{2^n} < \infty$ for $n \geq 1$. Therefore, 
by Riesz-Thorin Interpolation theorem and duality, we have $C_p < \infty$ for any $p \in (1,\infty)$. \hfill
$\qed$
\end{solution}

\bigskip

\begin{question}[4]
\hfill
\begin{figure}[h!]
  \centering
    \includegraphics[width=0.8\textwidth]{HA-4-4.png}
\end{figure}
\end{question}
\begin{solution} \hfill \\
In this solution, all integral are taken over the torus. We view $\phi * f$ as an operator, which we
denote as $T$.
We first show that the statement holds for the case $r=1$, and $r=q$. Suppose $r =1$, so $s=p$. By general
Minkowski inequality, we obtain
\eQb
||Tf||_{p} &=& (\int|\int \phi(x-y)f(y) dy|^p dx)^{\frac{1}{p}} \\
&\leq& \int(\int|\phi(x-y)|^p|f(y)|^p dx)^{\frac{1}{p}} dy = ||\phi||_{p}||f||_{1} < \infty,
\eQe
which implies that for $T:L^1 \to L^p$,
\eQb
||T||_{1 \to p} &\leq& ||\phi||_{p},
\eQe
thus bounded by assumption.
Now, suppose $r = q$, so $s = \infty$.
By Holder inequality, we obtain
\eQb
||Tf||_{\infty} &\leq& |\sup_{x \in \mathbb{T}} f(x)\phi(x-y)dy| \leq ||\phi f||_{1} 
\leq ||\phi||_{p}||f||_{q}, 
\eQe 
which implies that for $T:L^q \to L^{\infty}$,
\eQb
|||T||_{q \to \infty} \leq ||\phi||_{p},
\eQe
thus bounded by assumption. 

Hence, by Riesz-Thorin, we conclude that, for $1 \leq r \leq q$ and $T$ maps $L^r$ into $L^s$ with
\eQb
||T||_{r \to s} \leq ||T||_{1\to p}^{\theta} ||T||_{q \to \infty}^{1-\theta} \leq ||\phi||_{p},
\eQe
where $r^{-1} = \dfrac{\theta}{r_1} + \dfrac{1-\theta}{r_2}$, $s^{-1} = \dfrac{\theta}{s_1}
+ \dfrac{1-\theta}{s_2}$ with $(1,q)$ and $(p,\infty)$ being interpolation points respectively. \hfill
$\qed$
  
\end{solution}

\bigskip

\begin{question}[5]
\hfill
\begin{figure}[h!]
  \centering
    \includegraphics[width=0.8\textwidth]{HA-4-5.png}
\end{figure}
\end{question}
\begin{solution} \hfill \\
\textbf{(a)}
If $f = 0 \> \mu$-a.e in $X$, then by $(c)$, $[f] = 0$, so $f \in $ weak-$L^p$. 
Now, let $\gamma \in \mathbb{C}$, and $f \in $ weak-$L^p$. By $(b)$, we see that $\gamma f \in 
$ weak-$L^p$. Furthermore, let $f,g \in$ weak-$L^p$. By $(e)$, it follows that
\eQb
\big[f+g \big] \leq 2(\big[ f \big] + \big[ g \big]) < \infty,
\eQe 
which implies that $f+g \in $ weak-$L^p$. Therefore, the space is a linear space. All the
constituents will be proven in the later parts. 

\bigskip

\textbf{(b)} Let $\beta \in \mathbb{C}$. If $\beta = 0$, then $\big[ \beta f\big] 
= 0$, thus the equality holds trivially. Suppose $\beta \neq 0$. 
By the definition of weak-$L^p$, we have
\eQb
\big[ \beta f \big] &=& 
\sup_{\alpha > 0} 
\alpha \big[ \mu\{  |\beta f| > \alpha\} \big]^{\frac{1}{p}} 
=  |\beta| \sup_{\alpha > 0} 
\frac{\alpha}{|\beta|} \big[ \mu\{ |f| > \frac{\alpha}{|\beta|} \} \big]^{\frac{1}{p}} \\
&=& |\beta| \sup_{\alpha > 0} 
\alpha \big[ \mu\{ |f| > \alpha  \} \big]^{\frac{1}{p}} = |\beta|\big[ f\big] \\
\eQe

\bigskip

\textbf{(c)} Let $f = 0$  $\mu$-a.e. in $X$. By monotonicity of measure, we have
\eQb
\mu \{ |f| > \alpha \} = 0,
\eQe
for any $\alpha > 0$, which implies that $\big[ f \big]_{w,p} = 0$. Conversely, 
let $\big[ f \big]_{w,p} = 0$. As $\mathbb{C}$ is an integral domain, this implies that
for any $n \in \mathbb{N}$,
\eQb
\mu\{ |f| > \dfrac{1}{n} \} = 0,
\eQe 
so
\eQb
\mu\{ |f| > 0 \} = \mu \bigcup_{n=1}^{\infty} \{ |f| > \frac{1}{n} \}
\leq \sum_{n=1}^{\infty} \mu \{ |f| > \frac{1}{n} \} = 0. \\ 
\eQe
Therefore,
\eQb
\mu \{ |f| = 0 \} = 1,
\eQe
which completes the proof. 

\bigskip

\textbf{(d)} Let $X = \{ x_0, x_1 \}$, equipped with the uniform measure with $\mu(X) = 1$. 
Let $0 < a$, and $p \in [1,\infty)$. Define $f,g:X \to \mathbb{C}$ by
\eQb
f(x) &=& 
\left\{
    \begin{array}{ll}
        a  & \mbox{if } x = x_0 \\
        2a & \mbox{if } x = x_1, \\
    \end{array}
\right.
\eQe
and
\eQb
g(x) &=& 
\left\{
    \begin{array}{ll}
        k-a  & \mbox{if } x = x_0 \\
        k-2a & \mbox{if } x = x_1, \\
    \end{array}
\right.
\eQe
so 
\eQb
f+g(x) &=& 
\left\{
    \begin{array}{ll}
        k  & \mbox{if } x = x_0 \\
        k & \mbox{if } x = x_1, \\
    \end{array}
\right.
\eQe
with some $k > 2a$ to be determined. It follows that
\eQb
\big[ f \big] = 2a\frac{1}{2}^{\frac{1}{p}}, \> \big[ g \big] = k-2a
\>, \big[ f+g \big] = k,
\eQe 
for $k$ chosen large enough such that $(\frac{1}{2})^{\frac{1}{p}}(k-2a) \leq k-a$ holds. 
With the above equality, granted with the appropriate choice for $k$, dependent on $p$,
in order to violate the triangle inequality, we must have
\eQb
2a(\frac{1}{2})^{\frac{1}{p}} + (k-2a) < k.
\eQe
However, the above inequality is equivalent to
\eQb
2a((\frac{1}{2})^{\frac{1}{p}} -1) < 0,
\eQe
showing that the construction is valid for any $p \in [1,\infty)$.
We have shown that $\big[ \cdot \big]$ is not a norm in general. 

\bigskip

\textbf{(e)} Let $p \in [1,\infty)$. If $\big[ f\big] = \infty$ or $\big[ g \big] = \infty$, then
the inequality holds trivially. Suppose $f,g \in $ weak-$L^p$.
Observe that, for any $\alpha > 0$, 
\eQb
\mu \{|f+g| > \alpha\} \leq \mu\{ |f| + |g| > \alpha \} \leq \mu\{ |f| > \frac{\alpha}{2} \}
+ \mu\{ |g| > \frac{\alpha}{2}\}, 
\eQe
which implies that
\eQb
\big[f+g \big] &=& \sup_{\alpha > 0} \alpha \big[\lambda_{f+g}(\alpha)\big]^{\frac{1}{p}}
\leq
\sup_{\alpha > 0} \alpha \big( \big[\lambda_{f}(\frac{\alpha}{2})\big]^{\frac{1}{p}} + 
\big[\lambda_{g}(\frac{\alpha}{2})\big]^{\frac{1}{p}} \\
&\leq&  
2\big( \sup_{\alpha > 0} \alpha \big[\lambda_{f}(\alpha)\big]^{\frac{1}{p}} + 
\sup_{\alpha > 0} \alpha \big[\lambda_{g}(\alpha)\big]^{\frac{1}{p}}\big) = 2(\big[f\big] + \big[ g \big]), \\
\eQe
as required. Hence, we have shown that $\big[ \cdot \big]$ is a quasi-norm with $C \leq 2$. \hfill $\qed$  

\end{solution}

\bigskip

\begin{question}[6]
\hfill
\begin{figure}[h!]
  \centering
    \includegraphics[width=0.8\textwidth]{HA-4-6.png}
\end{figure}
\end{question}
\begin{solution} \hfill \\
\textbf{(a)} Let $0 < \alpha < \beta$. By monotonicity of measure, we have
\eQb
\mu\{|f| > \alpha \} &\geq& \mu\{ |f| > \beta \}.
\eQe
Hence, $\lambda_f$ is decreasing. Let $\{ \alpha_n \}$ be a sequence from $(\alpha,\infty)$
such that and $\alpha_n \to \alpha$ as $n \to \infty$.
Choose a subsequence from the sequence, which is strictly decreasing, 
denoted as $\{\alpha_{n_k}\}$. Now, by continuity of measure, it follows that
\eQb
\lim_{k \to \infty} \lambda_f(\alpha_{n_k}) &=& 
\lim_{k \to \infty} \mu\{ |f| > \alpha_{n_k} \} = \mu \bigcup_{k=1}^{\infty} \{|f| > \alpha_{n_k} \} \\ 
&=& \mu \{ |f| > \alpha \} = \lambda_f(\alpha). 
\eQe 
Hence, $\lambda_f(\alpha_n) \to \lambda_f(\alpha)$ as $n \to \infty$ and $\lambda_f$ is right
continuous.

\bigskip

\textbf{(b)}
Assume $f \in L^p$.
We have the following result from class:
\eQb
f \in L^p \implies ||f||_{p}^p = \int_{0}^{\infty} p\alpha^{p-1}\lambda_{f}({\alpha}) d\alpha.
\eQe
Therefore, by DCT, it follows that
\eQb
||f||_{p}^p &=& \int_{0}^{\infty} p\alpha^{p-1}\lambda_{f}({\alpha}) d\alpha \\
&=& \int_{0}^{\infty} \sum_{n=-\infty}^{\infty} p\alpha^{p-1}\lambda_{f}(\alpha)
\mathbf{1}_{(2^{n-1},2^n]}(\alpha) d\alpha \\ 
&=& \lim_{N \to \infty} \sum_{n=-N}^{N} \int_{0}^{\infty} p\alpha^{p-1} 
\lambda_{f}(\alpha)\mathbf{1}_{(2^{n-1},2^n]}(\alpha) d\alpha \\
&\geq& \lim_{N \to \infty} \sum_{n = -N}^{N} p(2^{n-1})^{p-1}\lambda_{f}(2^{n})2^{n-1}  
= \dfrac{p}{2^p} \sum_{n=-\infty}^{\infty} 2^{np}\lambda_{f}(2^n). 
\eQe
Hence, $\sum_{n=-\infty}^{\infty} 2^{np}\lambda_{f}(2^n) < \infty$. Conversely,
suppose that $\sum_{n=-\infty}^{\infty} 2^{np}\lambda_{f}(2^n) < \infty$. 
Set 
\eQb
A_n &=& = \{ 2^n \leq |f| \leq 2^{n+1} \}.
\eQe
By MCT, it follows that
\eQb
\sum_{n = -\infty}^{\infty} 2^{np} \lambda_{f}(2^n) &\leq& \sum_{n = -\infty}^{\infty}
2^{np} \mu(A_n) = 2^{-p}\lim_{N \to \infty} \int_{X} \sum_{n = -N}^{N}2^{(n+1)p}1_{A_n} d\mu \\
&=& 2^{-p} \int_{X} \sum_{n=-\infty}^{\infty} 2^{(n+1)p}1_{A_n} d\mu \\
&\leq& 2^{-p} \int_{X}|f|^p d\mu, \\
\eQe
which implies that $f \in L^p$.   

\bigskip

\textbf{(c)}
Let $p \in [1,\infty)$. Let $f$ be a simple function in $L^p$ such that $f = \sum_{i=1}^{k} a_i 
1_{E_i}$. As $L^p \subset \text{weak-}L^p$, for any $\alpha \in (0,\infty)$, we have 
$\lambda_{f}(\alpha) < \infty$. Furthermore, it follows that 
\eQb
\lambda_{f}(\alpha) = 0, \> \text{ if } \> \alpha \geq \text{max}\{ |a_i| \}, 
\eQe  
and, by right continuity of $\lambda_{f}$,
\eQb
\lambda_{f}(\alpha) = \lim_{\alpha' \to 0}\lambda_{f}(\alpha') \> \text{ if } \> \alpha < \text{min}\{ |a_i| \}.
\eQe
Without loss of generality, assume that $a_i \neq 0$ for all $i$.
By the above observation, it follows that
\eQb
\lim_{\alpha \to \infty} \alpha^{p} \lambda_{f}(\alpha) = 0 \> &\text{ and }& \>
\lim_{\alpha \to 0} \alpha^{p}\lambda_{f}(\alpha) = 0,
\eQe
so
\eQb
\lambda_{f}(\alpha) = o(\alpha^{-p})
\eQe
for both $a \to \infty$ and $a \to 0$ limits. Now, we use a standard density argument to show that
the claim holds true for any $f \in L^p$. Let $f \in L^p$. 
Fix $\epsilon > 0$. As simple functions are dense in $L^p$, 
there exists a simple function $s \in L^p$ such that 
\eQb
||f - s ||_{p} < \epsilon 
\eQe 
Then
\eQb
0 \leq \alpha^p\lambda_{f}(\alpha) &=& \alpha^p \lambda_{f-s}(\frac{\alpha}{2}) + 
\alpha^p\lambda_{s}(\frac{\alpha}{2}) \\ 
&\leq& 2^p ||f - s||_{p}^p + 2^p (\frac{\alpha}{2})^p \lambda_{s}(\frac{\alpha}{2}) < C\epsilon, 
\eQe
for some constant $C$ provided that $\alpha$ is small or large enough.
Therefore,
\eQb
\lambda_{f}(\alpha) = o(a^{-p}),
\eQe
for both $a \to \infty$ and $a \to 0$ limits as required. \hfill $\qed$
\end{solution}
\end{document}
